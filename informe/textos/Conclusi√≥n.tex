\section{Conclusión}


\textbf{¿Considera que las muestras obtenidas analizadas son representativas del comportamiento general de la red?} %Esto puede quedarse en esta sección%

\begin{description}
    \item[Red 1]: Es representativa. Había pocos dispositivos conectados y el uso se mantuvo constante a lo largo de toda la medición lo cuál resulto en una gran predominancia \texttt{IPv4} (mayormente con UDP como protocolo en el nivel superior) y pocos \texttt{ARP}, ya que no se conectaban nuevos dispositivos ni cambiaba la comunicación.
    
    \item[Red 2]: No es representativo porque se utilizó una aplicación que revisa qué dispositivos están conectados en la red wifi, y para ello emitió una gran cantidad de broadcasts consultando por las direcciones IP dentro de un rango.

    \item[Red 3]: Es representativo, ya que el uso que se le estaba dando a la red era el habitual en nuestra vivienda. Subida y bajada constante de datos es habitual por la profesión de los integrantes de esta red.
        
\end{description}

A excepción de unas pocas sorpresas, los resultados cayeron dentro de lo que esperabamos. Es clara la predominancia del protocolo \texttt{IPV4} en las tramas de cada red escuchada. Esto en parte es razonable, debido a la naturaleza de \texttt{ARP} (sólo es necesario cuando no se tiene una dirección MAC, una vez obtenida no se vuelve a solicitar con mucha frecuencia) y la baja adopción que tiene actualmente \texttt{IPv6}, aunque en el futuro cercano se volverá imprescindible. 


\newpage

\textbf{¿Hay alguna relación entre la entropía de las redes y alguna característica de las mismas (ej.:tamaño, tecnología, etc)?} %Esto tambien puede quedarse

Notamos que obtuvimos mediciones muy similares para las 3 redes y en base al análisis creemos que se debe justamente a las características que comparten: topologías (tipo estrella), usos (al menos durante la escucha) y tecnología (Wi-Fi).








